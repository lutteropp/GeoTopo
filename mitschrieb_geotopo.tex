\documentclass[a4paper,11pt,notitlepage]{report}

\usepackage{graphicx}
\usepackage[utf8]{inputenc}
\usepackage[T1]{fontenc}
\usepackage[ngerman]{babel}
\usepackage{bibgerm}
\usepackage{amsmath,amssymb,amsthm}
\usepackage{color}
\usepackage{enumerate}
\usepackage{tabularx}
\usepackage{subfig}
\usepackage{fancyhdr}
\usepackage{upgreek}
\usepackage[pdftex,pdfpagelabels,colorlinks,backref,pagebackref]{hyperref}
\usepackage{tikz} % SELBST HINZUGEFÜGT
% == Set the heading style ===================================================
\setlength{\headheight}{14pt}
\pagestyle{fancyplain}
\renewcommand{\chaptermark}[1]{\markboth{#1}{}}
\renewcommand{\sectionmark}[1]{\markright{\thesection\ #1}}
\lhead[\fancyplain{}{\thepage}]{\fancyplain{}{\rightmark}}
\rhead[\fancyplain{}{\leftmark}]{\fancyplain{}{\thepage}}
\cfoot{}
\renewcommand{\headrulewidth}{0.4pt}
% ============================================================================

% == Set correct values for fitting floats ===================================
\tolerance=2000
\emergencystretch=10pt

\setcounter{topnumber}{3}
\setcounter{totalnumber}{5}
\setcounter{bottomnumber}{2}

% To make those darn floats fit where they should
\setcounter{totalnumber}{9}
\setcounter{topnumber}{9}
\setcounter{bottomnumber}{9}
\renewcommand{\textfraction}{0.00}
\renewcommand{\topfraction}{1.0}
\renewcommand{\bottomfraction}{1.0}
% ============================================================================

% == German definitions for theorems etc. ==================================== 
\newtheorem{definition}{Definition}[chapter]
\newtheorem{theorem}{Satz}[chapter]
\newtheorem{lemma}{Lemma}[chapter]
\newtheorem{proposition}{Proposition}[chapter]
\newtheorem{corollary}{Korollar}[chapter]
\newtheorem{observation}{Beobachtung}[chapter]
\newtheorem{fact}{Fakt}[chapter]
\newtheorem{remark}{Bemerkung}[chapter]
\newtheorem{example}{Beispiel}[chapter]
% ============================================================================

% == Abkürzungen für die reellen, natürlichen, ganzen,... Zahlen =============
\newcommand{\R}{{\ensuremath{\mathbb{R}}}}
\newcommand{\N}{{\ensuremath{\mathbb{N}}}}
\newcommand{\Z}{{\ensuremath{\mathbb{Z}}}}
\newcommand{\C}{{\ensuremath{\mathbb{C}}}}
\newcommand{\Q}{{\ensuremath{\mathbb{Q}}}}
\newcommand{\F}{{\ensuremath{\mathbb{F}}}}
\newcommand{\Prim}{{\ensuremath{\mathbb{P}}}}
% ============================================================================

% == Makros für Autorenname und -adresse =====================================
\newcommand{\myaddress}[6]{%
  \parbox{\textwidth}{\textbf{\large #1}\\
    #2\\ #3\\ #4\\ 
    \ifthenelse{\equal{#5}{}}{}{Email: \href{mailto:#5}{\texttt{#5}}\\}
    \ifthenelse{\equal{#6}{}}{}{WWW: \href{#6}{\path|#6|}\\}
  } 
}

\newcommand{\myauthor}[1]{%
  \addtocontents{toc}{\protect\hspace{3.35ex}%
  \textsl{#1}\par}\vspace{-4ex}\quad\hfill\textsl{\Large #1}\vspace{8ex}}

\newcommand{\myname}[1]{\Large #1}

\title{\textbf{{Einführung in die Geometrie und Topologie - Mitschrieb -} \\[5ex] 
    {\Large Vorlesung im Wintersemester 2011/2012\\[5ex]}}}

%%%%%%%%%%%%%%%%%%%%%%%%%%%%%%%%%%%%%%%%%%%%%%%%%%
% Tragen Sie in der folg. Zeile Ihren Namen ein: %
%%%%%%%%%%%%%%%%%%%%%%%%%%%%%%%%%%%%%%%%%%%%%%%%%%
\author{\myname{Sarah Lutteropp}}


\newcommand{\OO}{{\ensuremath{\upsigma}}}

\begin{document}
\shorthandoff{"}
\maketitle
\setcounter{tocdepth}{1}
\tableofcontents

\section*{Vorwort}
Dies ist ein Mitschrieb der Vorlesung “Einführung in die Geometrie und Topologie” vom Wintersemester 2011/2012 am Karlsruher Institut für Technologie, die von Herrn Prof. Dr. Wilderich Tuschmann gehalten wird.

\chapter{Homotopie und Fundamentalgruppe}

\begin{definition}[Topologischer Raum]
Ein \underline{topologischer Raum} $X$ ist gegeben durch eine Menge $X$ und ein System $\OO$ von Teilmengen von $X$, den so genannten \underline{offenen Mengen} von $X$, welches unter beliebigen Vereinigungen und endlichen Durchschnitten abgeschlossen ist und $X$ und die leere Menge $\emptyset$ als Elemente enthält.
\newline
$X$ Menge, $\OO \subset \mathcal{P}(X) \colon$
\begin{itemize}
	\item (1) $O_1, O_2 \in \OO \Rightarrow O_1 \cap O_2 \in \OO$
	\item (2) $O_\alpha \in \OO, \alpha \in A, A \text{ Indexmenge} \Rightarrow \bigcup\limits_{\alpha \in A}{O_\alpha} \in \OO$
	\item (3) $X, \emptyset \in \OO$
\end{itemize}
\end{definition}

\begin{example}
\OO = $\{X, \emptyset\} \Rightarrow (X,\OO)$ ist topologischer Raum!
\end{example}

\begin{example}
$$X \text{ Menge, }\OO = \left\{\{x\}|x\in X\right\} + \text{Axiome, die zu erfüllen sind} \leadsto \tilde{\OO}$$
$\Rightarrow (X,\tilde{\OO})$ ist topologischer Raum.
$\OO$ ist "Basis" der Topologie $\tilde{\OO}$.
\end{example}

\begin{definition}[Metrischer Raum]
Ein \underline{metrischer Raum} $X$ ist eine Menge $X$ mit einer Abbildung $d \colon X \times X \rightarrow \R$, der \underline{"Metrik"} auf $X$, die folgende Eigenschaften erfüllt:

\begin{itemize}
	\item (1) $d(x,y) = d(y,x)$ \underline{"Symmetrie"}
	\item (2) $d(x,y) = 0 \Leftrightarrow x = y, d(x,y) \geq 0$ \underline{"Definitheit"}
	\item (3) $d(x,z) \leq d(x,y) + d(y,z)$ \underline{"Dreiecksungleichung"}
	\item $\forall x,y,x \in X$
\end{itemize}

\end{definition}

\begin{definition}[stetig]
Eine Abbildung $F \colon X \rightarrow Y$ zwischen topologischen Räumen $X$ und $Y$ heißt \underline{stetig}, falls die F-Urbilder offener Mengen in $Y$ offene Teilmengen von $X$ sind.
\end{definition}

\begin{remark}
Ist $(X,d)$ ein metrischer Raum, so sind die offenen Mengen der von der Metrik induzierten Topologie Vereinigungen von endlichen Durchschnitten von Umgebungen
$U_{\epsilon}(x):=\{y \in X | d (x,y) < \epsilon \} (\epsilon > 0)$, 
und $F \colon (X,d) \rightarrow (Y,d')$ ist stetig im obigen Sinn genau dann, falls für alle $\epsilon > 0$ ein $\delta > 0$ existiert mit $F(U_\delta (x)) \subset U_\epsilon (F(x))$.
\end{remark}

\begin{definition}[Homotopie]
Eine \underline{Homotopie} $H \colon f \simeq g$ zwischen zwei (stetigen) Abbildungen $f,g \colon X \rightarrow Y$ ist eine (stetige) Abbildung $$H \colon X \times I \footnote{$I = [0,1] \subset \R$} \rightarrow Y, (x,t) \mapsto H(x,t)$$ mit $H(x,0) = f(x) \text{ und } H(x,1) = g(x) \forall x \in X$.
\end{definition}

TODO:BILDER

\begin{remark}
$H$ heißt auch \underline{Homotopie} \underline{\underline{von $f$ nach $g$}}, eine solche ist also eine parametrisierte Schar von Abbildungen mit "Anfang" $f$ und "Ende" $g$. $f$ und $g$ heißen dann \underline{homotop}, in Zeichen: $f \simeq g$.
\end{remark}

\end{document}
